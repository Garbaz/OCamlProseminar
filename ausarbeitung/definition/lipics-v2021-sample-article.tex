\documentclass[a4paper,UKenglish,cleveref, autoref, thm-restate]{lipics-v2021}
%This is a template for producing LIPIcs articles. 
%See lipics-v2021-authors-guidelines.pdf for further information.
%for A4 paper format use option "a4paper", for US-letter use option "letterpaper"
%for british hyphenation rules use option "UKenglish", for american hyphenation rules use option "USenglish"
%for section-numbered lemmas etc., use "numberwithinsect"
%for enabling cleveref support, use "cleveref"
%for enabling autoref support, use "autoref"
%for anonymousing the authors (e.g. for double-blind review), add "anonymous"
%for enabling thm-restate support, use "thm-restate"
%for enabling a two-column layout for the author/affilation part (only applicable for > 6 authors), use "authorcolumns"
%for producing a PDF according the PDF/A standard, add "pdfa"

%\pdfoutput=1 %uncomment to ensure pdflatex processing (mandatatory e.g. to submit to arXiv)
\hideLIPIcs  %uncomment to remove references to LIPIcs series (logo, DOI, ...), e.g. when preparing a pre-final version to be uploaded to arXiv or another public repository

%\graphicspath{{./graphics/}}%helpful if your graphic files are in another directory

\bibliographystyle{plainurl}% the mandatory bibstyle

\title{Definition of Event Reordering and Data-races}

%\titlerunning{Dummy short title} %TODO optional, please use if title is longer than one line

\author{Tobias Hoffmann}{Albert-Ludwigs-Universität Freiburg, Germany}{garbaz@t-online.de}{}{}

\authorrunning{T. Hoffmann} %TODO mandatory. First: Use abbreviated first/middle names. Second (only in severe cases): Use first author plus 'et al.'

\Copyright{Tobias Hoffmann} %TODO mandatory, please use full first names. LIPIcs license is "CC-BY";  http://creativecommons.org/licenses/by/3.0/

\begin{CCSXML}
<ccs2012>
    <concept>
        <concept_id>10003752.10003753.10003761.10003762</concept_id>
        <concept_desc>Theory of computation~Parallel computing models</concept_desc>
        <concept_significance>500</concept_significance>
    </concept>
</ccs2012>
\end{CCSXML}
    
\ccsdesc[500]{Theory of computation~Parallel computing models}

\keywords{threads, data race} %TODO mandatory; please add comma-separated list of keywords

\category{} %optional, e.g. invited paper

\relatedversion{} %optional, e.g. full version hosted on arXiv, HAL, or other respository/website
%\relatedversiondetails[linktext={opt. text shown instead of the URL}, cite=DBLP:books/mk/GrayR93]{Classification (e.g. Full Version, Extended Version, Previous Version}{URL to related version} %linktext and cite are optional

%\supplement{}%optional, e.g. related research data, source code, ... hosted on a repository like zenodo, figshare, GitHub, ...
%\supplementdetails[linktext={opt. text shown instead of the URL}, cite=DBLP:books/mk/GrayR93, subcategory={Description, Subcategory}, swhid={Software Heritage Identifier}]{General Classification (e.g. Software, Dataset, Model, ...)}{URL to related version} %linktext, cite, and subcategory are optional

%\funding{(Optional) general funding statement \dots}%optional, to capture a funding statement, which applies to all authors. Please enter author specific funding statements as fifth argument of the \author macro.

\acknowledgements{}%optional

\nolinenumbers %uncomment to disable line numbering



%Editor-only macros:: begin (do not touch as author)%%%%%%%%%%%%%%%%%%%%%%%%%%%%%%%%%%
% \EventEditors{John Q. Open and Joan R. Access}
% \EventNoEds{2}
% \EventLongTitle{42nd Conference on Very Important Topics (CVIT 2016)}
% \EventShortTitle{CVIT 2016}
% \EventAcronym{CVIT}
% \EventYear{2016}
% \EventDate{December 24--27, 2016}
% \EventLocation{Little Whinging, United Kingdom}
% \EventLogo{}
% \SeriesVolume{42}
% \ArticleNo{0}
%%%%%%%%%%%%%%%%%%%%%%%%%%%%%%%%%%%%%%%%%%%%%%%%%%%%%%

\begin{document}

\maketitle

%TODO mandatory: add short abstract of the document
% \begin{abstract}
%     TODO: Abstract
% \end{abstract}

\section{Definition}
\label{sec:definition}

\paragraph*{Program \& Threads}
A \textit{concurrent program} $P$ consists of a set of \textit{threads} $T = \left\{ t_1, ..., t_n \right\}$.

\paragraph*{Action \& Event}

An \textit{event} $e = (i, a)$ is any \textit{action} $a$ performed by a thread $t_i$ which potentially influences or is influenced by the execution of other threads. An action can be:

\begin{itemize}
    \item $read(x)$ : Read from variable $x$
    \item $write(x)$ : Write to variable $x$
    \item $lock(l)$ : Acquire lock $l$
    \item $unlock(l)$ : Release lock $l$
    \item $fork(j)$ : Create child thread $t_j$
    \item $join(j)$ : Await child thread $t_j$ to terminate
\end{itemize}

Hereby is $x$ the identifier of a \textit{global variable}, $l$ the identifier of a \textit{lock} and $j$ is the index of a \textit{child thread} $t_j$.

\paragraph*{Trace of a program}

The \textit{trace} $\sigma = \left\{ e_1 , e_2 , ..., e_m \right\}$ of a program $P$ is the temporally ordered sequence of events of the threads $T$ in $P$ for a potential execution of $P$.

\paragraph*{Last-Write of a Read Action}

For a read event $e = (i, read(x))$, we define the \textit{last-write} of $e$ as the last event $e' = (j, write(x))$ that wrote to $x$ before $e$ by any thread. We write $lastwrite(e) := e'$.

\paragraph*{Reordering \& Happens-Before Relation}

The relation \textit{happens-before} between any two events $e=(i, a)$ and $e'=(j, a')$, written $e \prec e'$, is defined if any of the following conditions are met:

\begin{itemize}
    \item $i = j$ and $a$ happens before $a'$ in $t_i$
    \item $a = fork(j)$
    \item $a' = join(i)$
    \item $i \neq j$ and $a = unlock(l)$ and $a' = lock(l)$
\end{itemize}


A \textit{valid reordering} $r : \left\{1,...,m\right\} \rightarrow \left\{1,...,m\right\}$ of the trace $\sigma = \left\{e_1, ..., e_m\right\}$ of a program $P$ is a bijective function that preserves the \textit{happens-before} relation between all events:

$$
    \forall k \in \left\{1,...,m\right\} \forall l \in \left\{1,...,m\right\}: e_k \prec e_l \Rightarrow e_{r(k)} \prec e_{r(l)}
$$

\paragraph*{Data Race}

A \textit{data race} exists for a read event $e_k = (i, read(x))$ if there is a valid reordering $r$ of the program's trace $\sigma = \left\{e_1, ..., e_m\right\}$ that changes the last-read of it:

$$
    lastread(e_k) \neq lastread(e_{r(k)})
$$


%%
%% Bibliography
%%

%% Please use bibtex, 

% \bibliography{lipics-v2021-sample-article}

% \appendix

\end{document}
